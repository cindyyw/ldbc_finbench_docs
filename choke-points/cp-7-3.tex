\snbCPSection{7.3}{QEXE}{Execution of a transitive step}

This choke point tests the ability of the query execution engine to efficiently
execute transitive steps. Graph workloads may have transitive operations, for
example finding the shortest path between vertices. This involves repeated execution
of a short lookup, often on many values at the same time, while usually having
an end condition, \eg the target vertice being reached or having reached the border
of a search going in the opposite direction. For the best efficiency, these
operations can be merged or tightly coupled to the index operations themselves.
Also, parallelization may be possible but may need to deal with a global state,
\eg set of visited vertices. There are many possible tradeoffs between generality
and performance.

%%%%%%%%%%%%%%%%%%%%%%%%%%%%%%%%%%%%%%%%%%%%%%%%%%%%%%%%%%%%%%%%%%%%%%%%%%%%%%

\IfFileExists{choke-point-query-mapping/cp-7-3}{\snbCPSection{7.3}{QEXE}{Execution of a transitive step}

This choke point tests the ability of the query execution engine to efficiently
execute transitive steps. Graph workloads may have transitive operations, for
example finding a shortest path between vertices. This involves repeated execution
of a short lookup, often on many values at the same time, while usually having
an end condition, \eg the target vertice being reached or having reached the border
of a search going in the opposite direction. For the best efficiency, these
operations can be merged or tightly coupled to the index operations themselves.
Also parallelization may be possible but may need to deal with a global state,
\eg set of visited vertices. There are many possible tradeoffs between generality
and performance.

%%%%%%%%%%%%%%%%%%%%%%%%%%%%%%%%%%%%%%%%%%%%%%%%%%%%%%%%%%%%%%%%%%%%%%%%%%%%%%

\IfFileExists{choke-point-query-mapping/cp-7-3}{\snbCPSection{7.3}{QEXE}{Execution of a transitive step}

This choke point tests the ability of the query execution engine to efficiently
execute transitive steps. Graph workloads may have transitive operations, for
example finding a shortest path between vertices. This involves repeated execution
of a short lookup, often on many values at the same time, while usually having
an end condition, \eg the target vertice being reached or having reached the border
of a search going in the opposite direction. For the best efficiency, these
operations can be merged or tightly coupled to the index operations themselves.
Also parallelization may be possible but may need to deal with a global state,
\eg set of visited vertices. There are many possible tradeoffs between generality
and performance.

%%%%%%%%%%%%%%%%%%%%%%%%%%%%%%%%%%%%%%%%%%%%%%%%%%%%%%%%%%%%%%%%%%%%%%%%%%%%%%

\IfFileExists{choke-point-query-mapping/cp-7-3}{\snbCPSection{7.3}{QEXE}{Execution of a transitive step}

This choke point tests the ability of the query execution engine to efficiently
execute transitive steps. Graph workloads may have transitive operations, for
example finding a shortest path between vertices. This involves repeated execution
of a short lookup, often on many values at the same time, while usually having
an end condition, \eg the target vertice being reached or having reached the border
of a search going in the opposite direction. For the best efficiency, these
operations can be merged or tightly coupled to the index operations themselves.
Also parallelization may be possible but may need to deal with a global state,
\eg set of visited vertices. There are many possible tradeoffs between generality
and performance.

%%%%%%%%%%%%%%%%%%%%%%%%%%%%%%%%%%%%%%%%%%%%%%%%%%%%%%%%%%%%%%%%%%%%%%%%%%%%%%

\IfFileExists{choke-point-query-mapping/cp-7-3}{\input{choke-point-query-mapping/cp-7-3}}{}
}{}
}{}
}{}

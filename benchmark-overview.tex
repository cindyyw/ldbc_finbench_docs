\chapter{Benchmark Overview}
\label{sec:benchmark-overview}

\section{Practice basis}

The task force members design \ldbcfinbench with their rich practical experience in
financial industry based on a comprehensive survey of financial scenarios including
Risk Control, AML (Anti-Money Laundering), KYC (Know Your Customer), Stock Recommendation
and so on.

%%%%%%%%%%%%%%%%%%%%%%%%%%%%%%%%%%%%%%%%%%%%%%%%%%%%%%%%%%%%%%%%%%%%%%%%%%%%%%
%%%%%%%%%%%%%%%%%%%%%%%%%%%%%%%%%%%%%%%%%%%%%%%%%%%%%%%%%%%%%%%%%%%%%%%%%%%%%%
%%%%%%%%%%%%%%%%%%%%%%%%%%%%%%%%%%%%%%%%%%%%%%%%%%%%%%%%%%%%%%%%%%%%%%%%%%%%%%

\section{Design Concepts}

\ldbcfinbench is intended to be a credible, fair and representative benchmark.
It's designed with the following concepts:

\begin{itemize}
      \item \textbf{Based on real systems}. The task force members gathering
            together from industry and academia intend to design \ldbcfinbench
            to express and emphasize the special patterns of data and workload
            distinguished from other popular benchmarks. To do that,
            \ldbcfinbench is designed based on the rich practical experience of
            members and additional surveys.
      \item \textbf{Comprehensive and complete.} \ldbcfinbench is intended to
            cover most demands encountered in the management of complexly
            structured data in financial scenarios.
      \item \textbf{Challenging and instructive.} Benchmarks are known to direct
            product development in certain directions. \ldbcfinbench is informed
            by state-of-the-art in database research and industry practice
            to offer optimization challenges.
      \item \textbf{Easy to use and extendable.} As a benchmark offering value
            to many relevant stakeholders, \ldbcfinbench is designed to be easy
            to use. The effort for obtaining test results with it should be
            small.
      \item \textbf{Modularized.} \ldbcfinbench is broken into parts both in
            design and benchmark suite that can be individually addressed to
            stimulate innovation without imposing an overly high threshold for
            participation.
      \item \textbf{Reproducible and documented.} \ldbcfinbench is intended to
            specify the auditing rules and provide full disclosure reports of
            auditing of benchmark runs in accordance with the LDBC
            Bylaws~\cite{ldbc_byelaws}.
\end{itemize}


%%%%%%%%%%%%%%%%%%%%%%%%%%%%%%%%%%%%%%%%%%%%%%%%%%%%%%%%%%%%%%%%%%%%%%%%%%%%%%
%%%%%%%%%%%%%%%%%%%%%%%%%%%%%%%%%%%%%%%%%%%%%%%%%%%%%%%%%%%%%%%%%%%%%%%%%%%%%%
%%%%%%%%%%%%%%%%%%%%%%%%%%%%%%%%%%%%%%%%%%%%%%%%%%%%%%%%%%%%%%%%%%%%%%%%%%%%%%

\section{New features in FinBench}

\ldbcsnb, one of the earlier LDBC benchmarks, is modeled around the operation of a real social network site. It defines a data schema that represents a realistic social network including people and their activities during a period of time and also the workloads mimic the different usage scenarios found in operating a real social network site. Compared with \ldbcsnb, \ldbcfinbench is characterized by the special features and patterns of the data schema and queries that represent the characteristics of financial scenarios.

\subsection{Data Schema}

The data schema for \ldbcfinbench is designed to reflect the real data in the financial systems. Frequent
financial entities in real systems include accounts, medium, persons, companies, loans, etc. The
entities are vertices in the data schema while the edges reflect financial activities, \eg fund transferred
from one account to another. In their data schema, financial scenarios have these distinguished characteristics
compared to regular social networks.
\begin{itemize}
      \item Multiple edges can exist between two vertices, \eg Many transfer records exist between two accounts
      \item Dynamic attribute exists in vertex to mark entities status, \eg an account is marked as blocked
      \item Quantity attribute exists in edge, \eg Transfer edge has quantity attribute amount
\end{itemize}

The designed data schema is specified in \autoref{sec:data-definition}.

\subsection{Workloads}

In workloads and queries, financial scenarios have these distinguished characteristics.
\begin{itemize}
      \item More tight latency, \eg some queries need to return in less than 100ms.
      \item Write operations updating attributes, \eg marking an account as blocked.
      \item Recursive Path Filtering. Some queries filter data with backward dependency
            in variable-length paths, \eg finding all transfer paths A-[${e_1}$]->..-[${e_k}$]->B
            where the timestamp of each transfer edge ${e_i}$ in the path is larger than that of
            the previous ${e_{i-1}}$. In this pattern, the variable length path is qualified by
            the edge quantity attributes or the aggregation in the path, either along one path
            or a set of paths.
      \item Read-write Query, which is a query sequence with a mix of reads and writes reflecting the
            complexity of financial systems. Read-write query describes a desired pattern that risk control
            policies are checked, and corresponding actions are taken before financial activities like
            transfers are written down to storage. See \autoref{sec:read-write-query} for details.
      \item Truncation. In financial scenarios, the degree of hub vertex may reach million and even
            billion scales, especially when traversing on a graph. To handle the discordance between the tight
            latency requirements and power-law distribution of data in the system, truncation is introduced
            to reduce the complexity of queries. See \autoref{sec:truncation-on-hub-vertices} for details.
\end{itemize}

In \ldbcfinbench, there are two kinds of workloads:
\begin{itemize}
      \item Transaction Workload. It includes queries with a tight latency bound, which are usually
            queries hopping a few steps from a start vertice. There are complex reads, simple reads, write
            operations, and read-write queries in transaction workload. The Transaction Workload is specified
            in \autoref{sec:transaction-workload}.
      \item Analytics Workload. It is supposed to include more complicated queries, \eg triggers and pre-computed
            values in online systems. This part is future work that will be designed and discussed in the
            following versions. The Analytics Workload is specified in \autoref{sec:analytics-workload}.
\end{itemize}


%%%%%%%%%%%%%%%%%%%%%%%%%%%%%%%%%%%%%%%%%%%%%%%%%%%%%%%%%%%%%%%%%%%%%%%%%%%%%%
%%%%%%%%%%%%%%%%%%%%%%%%%%%%%%%%%%%%%%%%%%%%%%%%%%%%%%%%%%%%%%%%%%%%%%%%%%%%%%
%%%%%%%%%%%%%%%%%%%%%%%%%%%%%%%%%%%%%%%%%%%%%%%%%%%%%%%%%%%%%%%%%%%%%%%%%%%%%%

\section{Benchmark Workflow}

See \autoref{sec:auditing-rules} for the execution workflow of \ldbcfinbench.

%%%%%%%%%%%%%%%%%%%%%%%%%%%%%%%%%%%%%%%%%%%%%%%%%%%%%%%%%%%%%%%%%%%%%%%%%%%%%%
%%%%%%%%%%%%%%%%%%%%%%%%%%%%%%%%%%%%%%%%%%%%%%%%%%%%%%%%%%%%%%%%%%%%%%%%%%%%%%
%%%%%%%%%%%%%%%%%%%%%%%%%%%%%%%%%%%%%%%%%%%%%%%%%%%%%%%%%%%%%%%%%%%%%%%%%%%%%%
